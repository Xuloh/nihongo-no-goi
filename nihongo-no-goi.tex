\documentclass{article}

% classic packages
\usepackage[margin=60pt]{geometry}
\usepackage[colorlinks,linkcolor={black},citecolor={blue!80!black},urlcolor={blue!80!black}]{hyperref}

% setup japanese font
\usepackage[AutoFallBack=true]{xeCJK}
\setCJKmainfont{HanaMinA.ttf}[Path=./fonts/]
\setCJKfallbackfamilyfont{}{HanaMinB.ttf}[Path=./fonts/]

% setup furigana
\usepackage{ruby}
\renewcommand{\rubysep}{1pt}

% better tables
\usepackage[table]{xcolor}
\usepackage{array}
\usepackage{longtable}

% setup page headers and footers
\usepackage{fancyhdr}
\pagestyle{fancy}
\lhead{日本語の語彙}
\rhead{}
\cfoot{}
\newcommand{\thegit}{https://github.com/Xuloh/nihongo-no-goi}
\lfoot{\small Source disponible sur \url{\thegit}}
\rfoot{Page \thepage}

\begin{document}

\begin{center}
\Huge
\ruby{日}{に}\ruby{本}{ほん}\ruby{語}{ご}の\ruby{語}{ご}\ruby{彙}{い}
\end{center}

\rowcolors{1}{white}{gray!20}
\setlength{\extrarowheight}{23pt}

\section{Leçon 1}

\begin{longtable}{>{\huge}m{5cm} m{11cm}}

    \ruby{私}{わたし} / \ruby{私}{わたし}\ruby{達}{たち} & je, moi / nous \\
    \ruby{日}{に}\ruby{本}{ほん}\ruby{人}{じん} & Japonais (nationalité) \\
    \textasciitilde{}\ruby{人}{じん} & Nationalité de ce pays (ex : フランス人, la nationalité française) \\
    \ruby{何}{なに} & quoi \\
    \ruby{先}{せん}\ruby{生}{せい} & professeur, instituteur, maître (ne s'emploie pas pour désigner sa propre profession) \\
    \ruby{教}{きょう}\ruby{師}{し} & enseignant \\
    \ruby{学}{がく}\ruby{生}{せい} & étudiant \\
    \ruby{会}{かい}\ruby{社}{しゃ}\ruby{員}{いん} & employé d'une compagnie\\
    \ruby{社}{しゃ}\ruby{員}{いん} & employé (utilisé avec le nom de la compagnie, ex : IMCの社員) \\
    \ruby{銀}{ぎん}\ruby{行}{こう}\ruby{員}{いん} & employé de banque\\
    \ruby{医}{い}\ruby{者}{しゃ} & médecin \\
    \ruby{研}{けん}\ruby{究}{きゅう}\ruby {者}{しゃ} & chercheur \\
    \ruby{大}{だい}\ruby{学}{がく} & université \\
    \ruby{病}{びょう}\ruby{院}{いん} & hôpital \\
    だれ (どなた) & qui (version polie) \\
    \textasciitilde{}さい & \textasciitilde{} an(s) \\
    アメリカ & États-Unis \\
    イギリス & Angleterre \\
    フランス & France \\
    インド & Inde \\
    インドネシア & Indonésie \\
    \ruby{韓}{かん}\ruby{国}{こく} & Corée du Sud \\
    タイ & Thaïlande \\
    \ruby{中}{ちゅう}\ruby{国}{ごく} & Chine \\
    ドイツ & Allemagne \\
    \ruby{日}{に}\ruby{本}{ほん} & Japon \\
    ブラジル & Brésil \\

\end{longtable}

\section{Leçon 2}

\begin{longtable}{>{\huge}m{5cm} m{11cm}}

    これ / この \textasciitilde \newline それ / その \textasciitilde{} \newline あれ / あの \textasciitilde{} & ce, ceci / ce \textasciitilde{} ci (proche du locuteur) \newline ce, cela / ce \textasciitilde{} là (proche de l'interlocuteur) \newline ce, cela / ce \textasciitilde{} là-bas (éloigné) \\
        \ruby{本}{ほん} & livre \\
    \ruby{辞}{じ}\ruby{書}{しょ} & dictionnaire \\
    \ruby{雑}{ざっ}\ruby{誌}{し} & magazine \\
    \ruby{新}{しん}\ruby{聞}{ぶん} & journal \\
    ノート & cahier \\
    \ruby{手}{て}\ruby{帳}{ちょう} & agenda, carnet \\
    \ruby{名}{めい}\ruby{刺}{し} & carte de visite \\
    カード & carte \\
    \ruby{鉛}{えん}\ruby{筆}{ぴつ} & crayon papier \\
    ボールペン & stylo à bille \\
    {\LARGE シャープペンシル} & porte-mine \\
    \ruby{鍵}{かぎ} & clé \\
    \ruby{時}{と}\ruby{計}{けい} & montre, horloge, réveil \\
    \ruby{傘}{かさ} & parapluie \\
    \ruby{鞄}{かばん} & sac \\
    テレビ & télévision \\
    ラジオ & radio \\
    カメラ & appareil photo \\
    パソコン \newline (コンピューター) & ordinateur \\
    \ruby{車}{くるま} & voiture \\
    \ruby{机}{つくえ} & table, bureau \\
    \ruby{椅}{い}\ruby{子}{す} & chaise \\
    チョコレート & chocolat \\
    コーヒー & café \\
    (お)\ruby{土}{み}\ruby{産}{やげ} & souvenir, cadeau \\
    \ruby{日}{に}\ruby{本}{ほん}\ruby{語}{ご} & Japonais (langue) \\
    \ruby{英}{えい}\ruby{語}{ご} & Anglais (langue) \\
    \textasciitilde{}\ruby{語}{ご} & La langue de ce pays (ex : フランス語, la langue française) \\

\end{longtable}

\section{Leçon 3}

\begin{longtable}{>{\huge}m{5cm} m{11cm}}

    \ruby{教}{きょう}\ruby{室}{しつ} & salle de cours \\
    \ruby{食}{しょく}\ruby{堂}{どう} & restaurant, cantine \\
    \ruby{事}{じ}\ruby{務}{む}\ruby{所}{しょ} & bureau (salle) \\
    \ruby{会}{かい}\ruby{議}{ぎ}\ruby{室}{しつ} & salle de réunion \\
    \ruby{受}{うけ}\ruby{付}{つけ} & accueil \\
    ロビー & hall \\
    \ruby{部}{へ}\ruby{屋}{や} & salle \\
    トイレ\newline (お\ruby{手}{て}\ruby{洗}{あら}い) & toilettes \\
    \ruby{階}{かい}\ruby{段}{だん} & escalier \\
    エレベーター & ascenseur \\
    エスカレーター & escalator \\
    \ruby{自}{じ}\ruby{動}{どう}\ruby{販}{はん}\ruby{売}{ばい}\ruby{機}{き} & distributeur automatique \\
    \ruby{電}{でん}\ruby{話}{わ} & téléphone \\
    (お)\ruby{国}{くに} & pays \\
    \ruby{会}{かい}\ruby{社}{しゃ} & entreprise, société \\
    \ruby{家}{うち} & maison \\
    \ruby{靴}{くつ} & chaussures \\
    ネクタイ & cravate \\
    ワイン & vin \\
    \ruby{売}{う}り\ruby{場}{ば} & rayon (dans un magasin) \\
    \ruby{地}{ち}\ruby{下}{か} & sous-sol \\
    \textasciitilde{}\ruby{階}{かい} & \textasciitilde{} étage (1er étage est 2階 en japonais) \\
    \textasciitilde{}\ruby{円}{えん} & \textasciitilde{} yen \\
    いくら & combien \\
    イタリア & Italie \\
    スイス & Suisse \\
    ジャカルタ & Jakarta \\
    バンコク & Bangkok \\
    ベルリン & Berlin \\

\end{longtable}

\section{Leçon 4}

\begin{longtable}{>{\huge}m{5cm} m{11cm}}

    \ruby{起}{お}きます & se lever, se réveiller \\
    \ruby{寝}{ね}ます & dormir, se coucher \\
    \ruby{働}{はたら}きます & travailler \\
    \ruby{休}{やす}みます & se reposer, prendre un congé \\
    \ruby{勉}{べん}\ruby{強}{きょう}します & étudier \\
    \ruby{終}{お}わります & se terminer, finir \\
    デパート & grand magasin \\
    \ruby{銀}{ぎん}\ruby{行}{こう} & banque \\
    \ruby{郵}{ゆう}\ruby{便}{びん}\ruby{局}{きょく} & bureau de poste \\
    \ruby{図}{と}\ruby{書}{しょ}\ruby{館}{かん} & bibliothèque \\
    \ruby{美}{び}\ruby{術}{じゅつ}\ruby{館}{かん} & musée \\
    \ruby{今}{いま} & maintenant \\
    \textasciitilde{}\ruby{時}{じ} & \textasciitilde{} heure(s) \\
    \textasciitilde{}\ruby{分}{ふん} & \textasciitilde{} minute(s) \\
    \ruby{半}{はん} & demie (30 minutes) \\
    \ruby{午}{ご}\ruby{前}{ぜん} & matin, du matin, matinée \\
    \ruby{午}{ご}\ruby{後}{ご} & après-midi \\
    \ruby{朝}{あさ} & matin \\
    \ruby{昼}{ひる} & midi \\
    \ruby{晩}{ばん} / \ruby{夜}{よる} & soir / nuit \\
    おととい & avant-hier \\
    \ruby{昨日}{きのう} & hier \\
    \ruby{今日}{きょう} & aujourd'hui \\
    \ruby{明日}{あした} & demain \\
    \ruby{明後日}{あさって} & après-demain \\
    \ruby{今朝}{けさ} & ce matin \\
    \ruby{今}{こん}\ruby{晩}{ばん} & ce soir \\
    \ruby{休}{やす}み & repos, vacances, congé \\
    \ruby{昼}{ひる}\ruby{休}{やす}み & pause du midi \\
    \ruby{試}{し}\ruby{験}{けん} & examen \\
    \ruby{会}{かい}\ruby{議}{ぎ} & réunion (会議をします, faire une réunion) \\
    \ruby{映}{えい}\ruby{画}{が} & cinéma, film \\
    \ruby{毎}{まい} & tous (répétition, ex : 毎朝, tous les matins) \\
    \ruby{月}{げつ}\ruby{曜}{よう}\ruby{日}{び} & lundi \\
    \ruby{火}{か}\ruby{曜}{よう}\ruby{日}{び} & mardi \\
    \ruby{水}{すい}\ruby{曜}{よう}\ruby{日}{び} & mercredi \\
    \ruby{木}{もく}\ruby{曜}{よう}\ruby{日}{び} & jeudi \\
    \ruby{金}{きん}\ruby{曜}{よう}\ruby{日}{び} & vendredi \\
    \ruby{土}{ど}\ruby{曜}{よう}\ruby{日}{び} & samedi \\
    \ruby{日}{にち}\ruby{曜}{よう}\ruby{日}{び} & dimanche \\
    \textasciitilde{}から & à partir de \textasciitilde{} \\
    \textasciitilde{}まで & jusqu'à \textasciitilde{} \\
    と & et \\
    \ruby{番}{ばん}\ruby{号}{ご} & numéro \\
    ニューヨーク & New York \\
    \ruby{北}{ペ}\ruby{京}{キン} & Pékin/Beijing \\
    ロサンゼルス & Los Angeles \\
    ロンドン & Londres \\
\end{longtable}

\section{Métiers}

\begin{longtable}{>{\huge}m{5cm} m{11cm}}

    \ruby{公}{こう}\ruby{務}{む}\ruby{員}{いん} & fonctionnaire \\
    \ruby{駅}{えき}\ruby{員}{いん} & employé de gare \\
    \ruby{郵}{ゆう}\ruby{便}{びん}\ruby{局}{きょく}\ruby{員}{いん} & postier \\
    \ruby{店}{てん}\ruby{員}{いん} & vendeur \\
    \ruby{調}{ちょう}\ruby{理}{り}\ruby{師}{し} & cuisinier \\
    \ruby{理}{り}\ruby{容}{よう}\ruby{師}{し} / \ruby{美}{び}\ruby{容}{よう}\ruby{師}{し} & coiffeur pour hommes, barbier / coffeur, coiffeuse \\
    \ruby{弁}{べん}\ruby{護}{ご}\ruby{士}{し} & avocat (métier) \\
    \ruby{看}{かん}\ruby{護}{ご}\ruby{師}{し} & infirmier \\
    \ruby{運}{うん}\ruby{転}{てん}\ruby{手}{しゅ} & chauffeur \\
    \ruby{警}{けい}\ruby{察}{さつ}\ruby{官}{かん} & agent de police \\
    \ruby{外}{がい}\ruby{交}{こう}\ruby{官}{かん} & diplomate \\
    \ruby{政}{せい}\ruby{治}{じ}\ruby{家}{か} & homme politique \\
    \ruby{画}{が}\ruby{家}{か} & peintre \\
    \ruby{作}{さっ}\ruby{家}{か} & écrivain \\
    \ruby{音}{おん}\ruby{楽}{がく}\ruby{家}{か} & musicien \\
    \ruby{建}{けん}\ruby{築}{ちく}\ruby{家}{か} & architecte \\
    エンジニア & ingénieur \\
    デザイナー & designer, dessinateur, styliste \\
    ジャーナリスト & journaliste \\
    \ruby{歌}{か}\ruby{手}{しゅ} / \ruby{俳}{はい}\ruby{優}{ゆう} & chanteur / acteur, actrice \\
    スポーツ\ruby{選}{せん}\ruby{手}{しゅ} & sportif \\

\end{longtable}

\section{Nombres}

\begin{longtable}{>{\huge}m{5cm} m{11cm}}

    ゼロ & 0 \\
    \ruby{一}{いち} & 1 \\
    \ruby{二}{に} & 2 \\
    \ruby{三}{さん} & 3 \\
    \ruby{四}{よん/し} & 4 \\
    \ruby{五}{ご} & 5 \\
    \ruby{六}{ろく} & 6 \\
    \ruby{七}{なな/しち} & 7 \\
    \ruby{八}{はち} & 8 \\
    \ruby{九}{きゅう/く} & 9 \\
    \ruby{十}{じゅう} & 10 \\
    \ruby{百}{ひゃく} & 100 \\
    \ruby{千}{せん} & 1000 \\
    \ruby{一}{いち}\ruby{万}{まん} & 10 000 \\
\end{longtable}



\section{Autres}

\begin{longtable}{>{\huge}m{5cm} m{11cm}}

    \ruby{語}{ご}\ruby{彙}{い} & vocabulaire \\
    \ruby{食}{た}べます & manger \\
    \ruby{名}{な}\ruby{前}{まえ} & nom \newline (ex : 名前は何ですか?  Quel est votre nom ?) \\
    \ruby{元}{げん}\ruby{気}{き} & vitalité, énergie, entrain \newline (ex : 元気ですか?元気です。 Comment allez-vous ? Je vais bien.) \\
    \ruby{白}{しろ} / \ruby{白}{しろ}い & blanc (nom) / blanc (adjectif) \\
    \ruby{青}{あお} / \ruby{青}{あお}い & bleu (nom) / bleu (adjectif) \\
    \ruby{赤}{あか} / \ruby{赤}{あか}い & rouge (nom) / rouge (adjectif) \\
    \ruby{話}{はな}せます & pouvoir/savoir parler (ex : 日本語は話せます, je sais parler japonais) \\
    \ruby{野}{や}\ruby{菜}{さい} & légumes \\
    \ruby{魚}{さかな} & poisson \\
    \ruby{肉}{にく} & viande \\
    \ruby{漢}{かん}\ruby{字}{じ} & caractère chinois (kanji) \\

\end{longtable}

\end{document}
